%----------------------------------------------------------------------------------------
%	STAGE DESCRIPTION
%----------------------------------------------------------------------------------------
\section{Scopo dello stage}
\subsection{Scopo del progetto}
L'azienda AzzurroDigitale si occupa di trasformazione digitale nel mondo delle aziende manifatturiere con lo scopo di migliorarne e di implementare i loro processi grazie all'utilizzo di tecnologie innovative.\\

Lo studente verrà coinvolto nel progetto denominato “azzurra.flow” che diventerà un nuovo componente della piattaforma AWMS per il Workforce Management di AzzurroDigitale.
Seguendo un approccio “bottom-up” si vuole realizzare un assistente virtuale (veicolato tramite l’App Mobile ufficiale di AWMS) che abiliti, contemporaneamente, il lavoratore (dipendente) a percepire elementi di beneficio concreto, slegati quindi da concetti di spazio-tempo e l’Azienda a liberare risorse in ottica “win-win”.
Va da sé che il grado di adozione della soluzione, da parte del dipendente, sarà direttamente proporzionale alla percezione di beneficio determinando quindi la profondità del successo della situazione “win-win”.
Tramite “Azzurra” il dipendente potrà, ad esempio consultare il suo piano ferie/permessi, piuttosto che consultare la sua pianificazione in termini di turni di lavoro/postazione.
Lo studente entrerà a far parte del team di sviluppo ed avrà modo di acquisire competenze tecniche nell’utilizzo di linguaggi di programmazione e frameworks di sviluppo (Typescript, PHP, Angular, CakePHP, Node.js, Nest.js) di basi di dati (PostgreSQL, Redis) ma anche soft skills come comunicazione, collaborazione e utilizzo di metodologie Agili (SCRUM).

