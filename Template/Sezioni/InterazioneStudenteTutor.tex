%----------------------------------------------------------------------------------------
%	DESCRIPTION OF THE INTERACTION BETWEEN THE STUDENT AND THE INTERNAL TUTOR
%----------------------------------------------------------------------------------------
\subsection{Interazione tra studente e tutor aziendale}
% Personalizzare definendo le modalità di interazione col tutor aziendale

Lo stage verrà svolto nella sede dell'azienda rispettando le norme sul distanziamento sociale in vigore. Lo studente perciò sarà seguito da un team leader che lo aiuterà a integrasi in un team di sviluppo e di istruirlo sulle tecnologie e processi che verranno utilizzati durante lo stage. Inoltre il team leader monitorerà costantemente lo stato di avanzamento del progetto garantendo, il raggiungimento degli obiettivi fissati nella pianificazione, risolvendo prontamente eventuali problematiche riscontrate.

\subsubsection{Registro delle attività}
Per tracciare le attività svolte dallo studente durante lo stage è necessario riportare all'interno di un documento, condiviso con il tutor aziendale e con il tutor accademico, un riassunto di quanto svolto.
Lo strumento utilizzato è \textit{Google Drive} e al suo interno sarà presente una tabella contenente un riassunto chiaro e conciso delle attività svolte durante lo stage, riportate in ordine cronologico. \\
La registrazione delle attività viene eseguita rispettando le seguenti regole:
\begin{enumerate}
	\item Il documento deve essere aggiornato al termine di ogni giornata lavorativa;
	\item I dettagli del contenuto devono essere i seguenti:
		\begin{itemize}
			\item descrizione delle attività svolte;
			\item descrizione di eventuali problemi sorti.
		\end{itemize}
	\item Il tutor aziendale controllerà ogni nuovo inserimento, controllando che sia stato riportato correttamente. 
	\item Al termine dello stage, il tutor aziendale ed il tutor interno valideranno tale documento come garanzia dell'effettivo lavoro svolto.
\end{enumerate}

\pagebreak