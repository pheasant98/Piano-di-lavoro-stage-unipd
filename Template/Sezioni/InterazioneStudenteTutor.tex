%----------------------------------------------------------------------------------------
%	DESCRIPTION OF THE INTERACTION BETWEEN THE STUDENT AND THE INTERNAL TUTOR
%----------------------------------------------------------------------------------------

\subsubsection{Modalità di svolgimento}
In data XX/XX/2020 è stato fornito presso (https://www.unipd.it/stage) un modulo che permette la ripresa degli stage in presenza. Il modulo compilato dall’azienda ospitante verrà allegato al presente documento e inviato all’indirizzo stage@unipd.it.
Il suddetto stage si svolgerà ed inizierà in presenza dal XX/XX/2020. Per garantire la corretta formazione, lo studente sarà in contatto quotidiano con il team leader del gruppo di sviluppatori
nel quale sarà inserito. Il team leader, che sarà anche il tutor dello studente, si occuperà di
esplicitare i task che lo studente deve realizzare e gli obiettivi attesi nello svolgimento di ogni task.
Lo studente dovrà redigere un registro su cui, quotidianamente, segnalare le attività svolte. Tale
registro sarà validato anche dal suo tutor.

\subsection{Interazione tra studente e tutor aziendale}
% Personalizzare definendo le modalità di interazione col tutor aziendale

Le modalità di interazione tra studente e tutor aziendale ( Carlo Davanzo) previste sono:
	\begin{itemize}
		\item Daily meeting mattutino, della durata di circa 15 minuti, dove vengono discussi i task della giornata, ed eventuali problemi bloccanti;
		\item Weekly review dove vengono analizzate e discusse le attività che lo studente dovrà svolgere nella settimana successiva.
	\end{itemize}
Affiancamento con figura senior: durante la giornata lavorativa lo studente è affiancato da uno sviluppatore senior, che potrà aiutare e guidare lo studente nelle attività. Questa figura è anche responsabile della formazione “on the job”.

\pagebreak