\subsection{Contenuti formativi previsti}
% Personalizzare indicando le tecnologie e gli ambiti di interesse dello stage
Lo studente avrà modo di approfondire le seguenti tecnologie:
\begin{itemize}
	\item \textbf{Jira Software}: piattaforma di gestione progettuale utilizzata in azienda;
	\item \textbf{Jira Confluence}: piattaforma di gestione documentale utilizzata in azienda;
	\item \textbf{Node.js}: runtime di JavaScript Open source multipiattaforma orientato agli eventi per l’esecuzione di codice JavaScript, costruita sul motore JavaScript V8 di Google Chrome;
	\item \textbf{Nest.js}: framework per la creazione di applicazioni lato server Node.js efficienti e scalabili. Utilizza Javascript progressivo, è costruito e supporta pienamente TypeScript e combina elementi di OOP (Object Oriented Programming), FP (Functional Programming) e FRP (Functional Reactive Programming);
	\item \textbf{Angular}: framework per la creazione di applicazioni lato client con architettura MVC (Model View Controller) e Model–view–viewmodel (MVVM) insieme a componenti comunemente usate nelle applicazioni RIA;
	\item \textbf{Ionic}: framework per lo sviluppo di applicazioni mobili con HTML, CSS e JavaScript;
	\item \textbf{PHP}: linguaggio di scripting interpretato principalmente utilizzato per sviluppare applicazioni web lato server;
	\item \textbf{CakePHP}: framework per la realizzazione di applicazioni web, scritto in PHP;
	\item \textbf{PostgreSQL}: DBMS ad oggetti rilasciato con licenza libera.
\end{itemize}
\newpage