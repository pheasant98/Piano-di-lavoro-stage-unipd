%----------------------------------------------------------------------------------------
%   USEFUL COMMANDS
%----------------------------------------------------------------------------------------

\newcommand{\dipartimento}{Dipartimento di Matematica ``Tullio Levi-Civita''}

%----------------------------------------------------------------------------------------
% 	USER DATA
%----------------------------------------------------------------------------------------

% Data di approvazione del piano da parte del tutor interno; nel formato GG Mese AAAA
% compilare inserendo al posto di GG 2 cifre per il giorno, e al posto di 
% AAAA 4 cifre per l'anno
\newcommand{\dataApprovazione}{Data}

% Dati dello Studente
\newcommand{\nomeStudente}{Federico}
\newcommand{\cognomeStudente}{Perin}
\newcommand{\matricolaStudente}{1170747}
\newcommand{\emailStudente}{federico.perin.1@studenti.unipd.it}
\newcommand{\telStudente}{+ 39 342 193 6773}

% Dati del Tutor Aziendale
\newcommand{\nomeTutorAziendale}{Carlo}
\newcommand{\cognomeTutorAziendale}{Davanzo}
\newcommand{\emailTutorAziendale}{carlo.davanzo@azzurrodigitale.com}
\newcommand{\telTutorAziendale}{+39 347 680 4833}
\newcommand{\ruoloTutorAziendale}{Team leader}

% Dati dell'Azienda
\newcommand{\ragioneSocAzienda}{AzzuroDigitale}
\newcommand{\indirizzoAzienda}{Via della Croce Rossa, 42 – Padova Italy}
\newcommand{\sitoAzienda}{https://www.azzurrodigitale.com/}
\newcommand{\emailAzienda}{}
\newcommand{\partitaIVAAzienda}{}

% Dati del Tutor Interno (Docente)
\newcommand{\titoloTutorInterno}{Prof.}
\newcommand{\nomeTutorInterno}{Claudio Enrico}
\newcommand{\cognomeTutorInterno}{Palazzi}

\newcommand{\prospettoSettimanale}{
     % Personalizzare indicando in lista, i vari task settimana per settimana
     % sostituire a XX il totale ore della settimana
     \subsubsection{Prima Settimana 01/07-03/07 (24 ore)}   
	     \begin{itemize}
	     	\item \textbf{Formazione Angular}: corso Udemy + review di alcuni componenti di AWMS;
	     	\item \textbf{Formazione Ionic}: corso Udemy + review di alcuni componenti di AWMS Azzurra (mobile app).
	     \end{itemize}  
    \subsubsection{Seconda Settimana 06/07-10/07 (40 ore)}
    	 \begin{itemize}
    		\item \textbf{Formazione NestJS}: corso Udemy + review di alcuni componenti di “Azzurra” già sviluppati;
    		\item \textbf{Unit testing}: (Jasmine+Karma) lato frontend;
    		\item \textbf{End-to-end testing}: (Appium+Cucumber.js) lato mobile app.
    	\end{itemize}
    
	\subsubsection{Terza Settimana 13/07-17/07 (40 ore)}
		 \begin{itemize}
			\item Approfondimenti architetture a micro-services e loro implementazione in AWMS Platform;
			\item Analisi implementazione di un conversational flow editor visuale;
			\item Software selection (con test/poc) per lo sviluppo di un conversational flow editor visuale.
		\end{itemize}	
	
    \subsubsection{Quarta Settimana 20/07-24/07  (40 ore)}
    	 \begin{itemize}
	    	\item Contributi alla redazione della Solution Design di “Azzurra”;
	    	\item Contributi alla documentazione sorgenti di “Azzurra” (frontend/backend).
    	\end{itemize}
    	
    \subsubsection{Quinta Settimana 27/07-31/07 (40 ore)}
    	 \begin{itemize}
    			\item \textbf{Flow-engine (Azzurra.flow)}: corso Udemy + review di alcuni componenti di AWMS;
    			\item Aspetti di scalabilità di un flow-engine (concorrenzialità, HA, persistenza/storicizzazione
    			messaggi)
    	\end{itemize}
    	
    \subsubsection{Sesta Settimana 03/08-07/08 (40 ore)}
     	 \begin{itemize}
     		\item Contributi alla redazione della Solution Design di “Azzurra”;
     		\item Implementazione Push Notifications (lato mobile App);
     		\item Implementazione Push Notifications (lato backend).
     	\end{itemize}
    
    \subsubsection{Settima Settimana 17/08-21/08 (40 ore)}
    	 \begin{itemize}
    		\item Progettazione e documentazione template engine multi-lingua;
    		\item Implementazione template engine multi-lingua (l’assistente virtuale dovrà avere il supporto multi-lingua) basato su sintassi “mustache”.
    	\end{itemize}	
    
    \subsubsection{Ottava Settimana 24/08-28/08 (40 ore)}
    	 \begin{itemize}
    		\item Gestione comportamenti mobile app in condizioni di mancanza di connettività (corner cases, messaggi di feedback, landing pages).
    	\end{itemize}
    
    \subsubsection{Nona Settimana 31/08-01/09 (16 ore)}
	    \begin{itemize}
	    	\item Continuazione ottava settimana.
	    \end{itemize}
	  %	\paragraph{}
    %	\paragraph*{Obiettivi} \mbox{}\\ [1mm]
    %	L'obiettivo fissato in questo periodo è il seguente:
    %	\begin{itemize}
    %		\item \textbf{OB-X}: .
    %	\end{itemize}
	 %   \paragraph*{Descrizione} \mbox{}\\ [1mm]
        
}

% Indicare il totale complessivo (deve essere compreso tra le 300 e le 320 ore)
\newcommand{\totaleOre}{}

\newcommand{\obiettiviObbligatori}{
	 \item \textbf{OB-1}: competenza nello sviluppo delle singole attività identificate con i linguaggi PHP, Typescript.
}

\newcommand{\obiettiviDesiderabili}{
	 \item \textbf{OD-1}: capacità autonoma di analisi delle singole attività delle soluzioni tecniche viste durante il progetto;
	 \item \textbf{OD-2}: capacità autonoma di progettazione delle singole attività delle soluzioni tecniche viste durante il progetto.
}
