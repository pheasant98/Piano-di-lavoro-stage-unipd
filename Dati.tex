%----------------------------------------------------------------------------------------
%   USEFUL COMMANDS
%----------------------------------------------------------------------------------------

\newcommand{\dipartimento}{Dipartimento di Matematica ``Tullio Levi-Civita''}

%----------------------------------------------------------------------------------------
% 	USER DATA
%----------------------------------------------------------------------------------------

% Data di approvazione del piano da parte del tutor interno; nel formato GG Mese AAAA
% compilare inserendo al posto di GG 2 cifre per il giorno, e al posto di 
% AAAA 4 cifre per l'anno
\newcommand{\dataApprovazione}{Data}

% Dati dello Studente
\newcommand{\nomeStudente}{Federico}
\newcommand{\cognomeStudente}{Perin}
\newcommand{\matricolaStudente}{1170747}
\newcommand{\emailStudente}{federico.perin.1@studenti.unipd.it}
\newcommand{\telStudente}{+ 39 342 193 6773}

% Dati del Tutor Aziendale
\newcommand{\nomeTutorAziendale}{Carlo}
\newcommand{\cognomeTutorAziendale}{Davanzo}
\newcommand{\emailTutorAziendale}{}
\newcommand{\telTutorAziendale}{}
\newcommand{\ruoloTutorAziendale}{Team leader}

% Dati dell'Azienda
\newcommand{\ragioneSocAzienda}{AzzuroDigitale}
\newcommand{\indirizzoAzienda}{Via della Croce Rossa, 42 – Padova Italy}
\newcommand{\sitoAzienda}{https://www.azzurrodigitale.com/}
\newcommand{\emailAzienda}{}
\newcommand{\partitaIVAAzienda}{}

% Dati del Tutor Interno (Docente)
\newcommand{\titoloTutorInterno}{Prof.}
\newcommand{\nomeTutorInterno}{Claudio Enrico}
\newcommand{\cognomeTutorInterno}{Palazzi}

\newcommand{\prospettoSettimanale}{
     % Personalizzare indicando in lista, i vari task settimana per settimana
     % sostituire a XX il totale ore della settimana
     \subsubsection{Prima Settimana}
    	\paragraph{}
    	\paragraph*{Obiettivi} \mbox{}\\ [1mm]
    	Gli obiettivi fissati in questo periodo sono i seguenti:
        \begin{itemize}
            \item \textbf{OB-X}: .
         %   \item \textbf{OB-2}: .
        \end{itemize}
	    \paragraph*{Descrizione} \mbox{}\\ [1mm]
    	
    \subsubsection{Seconda Settimana}
    	\paragraph{}
    	\paragraph*{Obiettivi} \mbox{}\\ [1mm]
    	Gli obiettivi fissati in questo periodo sono i seguenti:
    	\begin{itemize}
    		\item \textbf{OB-X}: .
    	%	\item \textbf{OB-4}: .
    	\end{itemize}
        \paragraph*{Descrizione} \mbox{}\\ [1mm]
    
	\subsubsection{Terza Settimana}
		\paragraph{}
		\paragraph*{Obiettivi} \mbox{}\\ [1mm]
		Gli obiettivi fissati in questo periodo sono i seguenti:
		\begin{itemize}
			\item \textbf{OB-X}: .
		%	\item \textbf{OF-2}: .
		\end{itemize}
	%	Questi obiettivi sono facoltativi in quanto è necessaria la presenza in azienda per poterli perseguire ma, a causa del distanziamento sociale dovuto allo stato pandemico attuale, non è garantita. Si prevede dunque una pianificazione alternativa data dall'approfondimento dei seguenti obiettivi:
	%	\begin{itemize}
		%	\item \textbf{OB-2}: ;
		%	\item \textbf{OB-4}: .
	%	\end{itemize}
		\paragraph*{Descrizione} \mbox{}\\ [1mm]
	
    \subsubsection{Quarta Settimana}
    	\paragraph{}
    	\paragraph*{Obiettivi} \mbox{}\\ [1mm]
    	L'obiettivo fissato in questo periodo è il seguente:
        \begin{itemize}
            \item \textbf{OB-X}: .
        \end{itemize}
    	
	    \paragraph*{Descrizione} \mbox{}\\ [1mm]
    	
    \subsubsection{Quinta Settimana}
    	\paragraph{}
    	\paragraph*{Obiettivi} \mbox{}\\ [1mm]
    	L'obiettivo fissato in questo periodo è il seguente:
        \begin{itemize}
            \item \textbf{OB-X}: .
        \end{itemize}
	    \paragraph*{Descrizione} \mbox{}\\ [1mm]
    	
     \subsubsection{Sesta Settimana}
     	\paragraph{}
     	\paragraph*{Obiettivi} \mbox{}\\ [1mm]
		L'obiettivo fissato in questo periodo è il seguente:
        \begin{itemize}
            \item \textbf{OB-X}: .
        \end{itemize}
	    \paragraph*{Descrizione} \mbox{}\\ [1mm]
    
    \subsubsection{Settima Settimana}
    	\paragraph{}
    	\paragraph*{Obiettivi} \mbox{}\\ [1mm]
    	L'obiettivo fissato in questo periodo è il seguente:
        \begin{itemize}
            \item \textbf{OB-X}: .
        \end{itemize}
	    \paragraph*{Descrizione} \mbox{}\\ [1mm]
    
    \subsubsection{Ottava Settimana}
    	\paragraph{}
    	\paragraph*{Obiettivi} \mbox{}\\ [1mm]
    	L'obiettivo fissato in questo periodo è il seguente:
    	\begin{itemize}
    		\item \textbf{OB-X}: .
    	\end{itemize}
	    \paragraph*{Descrizione} \mbox{}\\ [1mm]
        
}

% Indicare il totale complessivo (deve essere compreso tra le 300 e le 320 ore)
\newcommand{\totaleOre}{}

\newcommand{\obiettiviObbligatori}{
	 \item \textbf{OB-1}: ;
	% \item \textbf{OB-2}: implementazione di una skill basata su un intent per Google Assistant, sottoforma di proof of concept;
 	% \item \textbf{OB-3}: studio e analisi di Amazon Alexa;
	% \item \textbf{OB-4}: implementazione di una skill basata su un intent per Amazon Alexa, sottoforma di proof of concept; 
	% \item \textbf{OB-5}: stesura analisi comparativa delle tecnologie studiate;
	% \item \textbf{OB-6}: studio dell'algoritmo HPP (Hidden Probabilistic Parser);
	% \item \textbf{OB-7}: realizzazione di una grammatica che implementi una skill non banale, integrata al sistema Zucchetti esistente.
}

\newcommand{\obiettiviDesiderabili}{
	 \item \textbf{OD-1}: ;
}

\newcommand{\obiettiviFacoltativi}{
 	 \item \textbf{OF-1}: ;
	% \item \textbf{OF-2}: implementazione di una skill basata su intent per Apple Siri, sottoforma di proof of concept.
}